% Template by Grant Lemons

\documentclass{lab}

% Document Info
% Your Name
\newcommand{\name}{  }
% Provide a concise title of the investigation that captures the relationship between the independent and dependent variable.
\newcommand{\reporttitle}{  }

\begin{document}
	% Change the name to one of the two not both
        \section*{Introduction / Background}
	% Write your introduction/background in plaintext here

	\subsection*{Research Questions}
	% ===========================
	% Probably only needed for IA
	% ===========================
	% State in 1-2 sentences a focused research question that captures the essence of the investigation.
	% The question should be cast using your independent and dependent variable.
	% For instance: “How does changing X (your independent variable), impact or affect Y (your dependent variable)."
	This lab will investigate the following research questions.
        \begin{itemize}
                \vspace{-0.5cm}
                \itemsep-3pt
		% Example questions
                \item \textbf{How does the the radius of the rotational motion effect the rotational speed of the bob holding other variables constant?}
                \item \textbf{How does the mass of the bob effect the rotational speed of the bob holding other variables constant?}
                \item \textbf{How does the mass of the counterweight effect the rotational speed of the bob holding other variables constant?}
        \end{itemize}

	% Hypothesis	
	% ===========================
	% Probably only needed for IA
	% ===========================
	% Craft a well-written sentence that summarizes your hypothesis (i.e. what you think will happen), what you expect the outcome to be, and why you believe what you do using scientific terms and logic.
	% For instance: “I think that by changing X (your independent variable), then Y (the dependent variable) should be increase or decrease because (insert explanation)."

	% Variables Table
	% ===========================
	% Probably only needed for IA
	% ===========================
	% Clearly state the dependent, independent and control variables in a Variable Table.
	% Minimum number of 5 IV changes X 3 trials each must be measured. Average of a minimum of three trials.
	\noindent\begin{minipage}{\textwidth}
        \subsection*{Variables}
        \vspace{-0.2cm}
	\renewcommand{\arraystretch}{1.1}
        \begin{table}[H]
            \centering
            \begin{tabular}{l|l|l|l}
		    \dtoprule
		    \textbf{Variable type}	& \textbf{Variable}	& \textbf{Details}	& \textbf{Units \& Uncertainty}	\\
		    Independent Variable	& Var to be changed	& How will it be changed& Var units and uncertainty	\\
		    Dependent Variable		& Var to be changed	& How will it be changed& Var units and uncertainty	\\
		    Control Variables		& Var to be changed	& How will it be changed& Var units and uncertainty	\\
		    % Fill these in
		    % Odds are you won't need these for a normal lab report
	    	    \dbottomrule
	    \end{tabular}
	    \caption{Table 1 - Variables}
	    \label{table: variables}
	\end{table}
	\end{minipage}

	% Material & Equipment List
	% Simple version
	\section*{Materials}
        \begin{itemize}
                \vspace{-0.3cm}
                \itemsep-3pt
                \item Item 1
		\item Item 2
		\item Item 3
        \end{itemize}

	% IA version
	\noindent\begin{minipage}{\textwidth}
        \vspace{-0.2cm}
        \renewcommand{\arraystretch}{1.1}
        \begin{table}[H]
            \centering
            \begin{tabular}{l|l|l|l}
                    \dtoprule
                    \textbf{Material}	& \textbf{Quantity Measured}	& \textbf{Absolute Uncertainty}	& \textbf{Percent Uncertainty}	\\
		    Material 1		& $\SI{5.00}{\cubic\cm}$	& $\pm\SI{0.005}{\cubic\cm}$	& 0.1\%				\\
		    Material 2		& $\SI{2.00}{\gram}$		& $\pm\SI{0.005}{\gram}$	& 0.2\%				\\
		    Material 3		& $\SI{10.0}{\kilo\gram}$	& $\pm\SI{0.01}{\kilo\gram}$	& 0.3\%				\\
                    % Fill these in
                    \dbottomrule
            \end{tabular}
            \caption{Table 2 - Material / Equipment List}
            \label{table: materials}
        \end{table}
        \end{minipage}

	% Safety
	% ===========================
	% Probably only needed for IA	
	% This is more of a chemistry thing
	% ===========================
	% List and describe all safety, environmental and ethical issues in the investigation.  Chemistry students must list and specifically address the hazards of all chemicals used in an investigation.
	\noindent\begin{minipage}{\textwidth}
        \subsection*{Safety}
        \vspace{-0.2cm}
        \renewcommand{\arraystretch}{1.1}
        \begin{table}[H]
            \centering
            \begin{tabular}{l|l|l|l}
                    \dtoprule
		    \textbf{Substance}  & \textbf{Safety}		& \textbf{Environmental}		& \textbf{Ethics}			\\
                    Substance 1          & known hazards		        & disposal and environmental impact	& how to deal with danger	\\
                    
                    Substance 2          & known hazards		        & disposal and environmental impact	& how to deal with danger	\\
		    % Fill these in
                    \dbottomrule
            \end{tabular}
	    \caption{Table 3 - Safety, Evironmental, and Ethical Considerations}
            \label{table: safety}
        \end{table}
        \end{minipage}
	
	% Experimental Design
	% ===========================
	% Probably only needed for IA
	% ===========================
	% Put an image in the same directory and label it
	% Fill in path to image below
	\begin{figure}[H]
		\includegraphics[width=\textwidth, center]{  }
		\caption{Figure 1 - Experimental setup showing blah blah blah}
		\label{fig: setup}
	\end{figure}

	% Procedure
	\section*{Procedure}
	\begin{enumerate}
                \vspace{-0.3cm}
                \itemsep-3pt
                \item Detailed, in depth description of a step in the process
		\item Use “useful words”  to capture the description: Attach, Weigh, Measure, Pour, Record, Connect, Join, Calculate, Determine, Launch, Read, Mark, Label, etc.
        \end{enumerate}

	% Raw Data table
	\noindent\begin{minipage}{\textwidth}
        \section*{Raw Data}
        \vspace{-0.2cm}
	\renewcommand{\arraystretch}{1.1}
        \begin{table}[H]
            \centering
            \begin{tabular}{l|l|l|l}
		    \dtoprule
		    % The styling of this depends on your data
		    % You just kind of need to figure it out
		    % The ampersand (&) seperates values on the same row
		    % The letters (in this case l|l|l|l) show how many rows there are and the alignment (left)
		    % Use \\ at the end of every row
	    	    \dbottomrule
	    \end{tabular}
	    \caption{Table 4 - Raw Data}
	    \label{table: raw-data}
	\end{table}
	\end{minipage}

	% Analysis and Calculations
	\section*{Calculations}
	% For equasions on their own line use this notation \[\] with the equasions in between the [ and the \]
	% For more info on LaTeX equasions look it up, theres probably a command for what you need
	% For instance here is the equasion for standard deviation
	\[\sigma = \sqrt{\dfrac{\Sigma(x_i-\mu)^2}{N}}\]

	% Results / Processed Data table	
	\noindent\begin{minipage}{\textwidth}
        \section*{Calculated Data}
        \vspace{-0.2cm}
	\renewcommand{\arraystretch}{1.1}
        \begin{table}[H]
            \centering
            \begin{tabular}{l|l|l|l}
		    \dtoprule
		    % The styling of this depends on your data
		    % You just kind of need to figure it out
		    % The ampersand (&) seperates values on the same row
		    % The letters (in this case l|l|l|l) show how many rows there are and the alignment (left)
		    % Use \\ at the end of every row
	    	    \dbottomrule
	    \end{tabular}
	    \caption{Table 5 - Calculated Data}
	    \label{table: calculated-data}
	\end{table}
	\end{minipage}

	% Graphs
	% Although graphs can be made in LaTeX, you probably don't want to
	% Export a graph from excel as PNG, put it in this folder, and plug in the file name here
	\begin{figure}[H]
		\includegraphics[width=\textwidth, center]{  }
		\caption{Figure 2 - Graph of Blah vs Blah}
		\label{fig: graph-1}
	\end{figure}

	% Analysis
	\section*{Analysis}
	% Identify the errors that impacted your results and list them one by one in a table. Classify each error as systematic or random error.
	% Indicate how each error affected your results – if at all. Avoid being superficial.
	
	% =======================
	% This can probably be a paragraph for a normal lab report but should be a table for the IA
	% This is one of the most important parts of the lab
	% =======================

	% Example provided below
	\noindent\begin{minipage}{\textwidth}
        \vspace{-0.2cm}
	\renewcommand{\arraystretch}{1.1}
        \begin{table}[H]
            \centering
            \begin{tabular}{L{1.445in}|L{1.445in}|L{1.445in}|L{1.445in}}
            \toprule
            \textbf{Error or Problem}    & \textbf{Type of Error}    & \textbf{Error's Effect}    & \textbf{Suggestion for Improvement}\\
            \hline
            Artificial increase in recorded temperature from stirring the thermometer & Systematic & The change in temperature calculated was less than it should have been & Be consistent with stirring for each temperature recorded \\
            \hline
            Placement of the thermometer & Random & Recorded temperature does not reflect the average temperature & Stir the water while it heats, such that the heated water is distributed more evenly \\
            \hline
            Different initial temperatures & Systematic & The change in temperature in the second trial was less than it should have been & Use new, room temperature, water for each trial \\
            \bottomrule
        \end{tabular}
	\caption{Table 6 - Errors \& Suggestions}
	    \label{table: errors}
	\end{table}
	\end{minipage}

	\section*{Conclusion}
	% Use your data and the results to formulate, summarize and support a conclusion.
	% Use evidence-based reasoning in your writing by referencing your data, results tables and graphs as proof of what is claimed.
	% Your conclusion(s) should be clearly related to the research question and the purpose and stated hypothesis.
	% If a numerical result is the object of the lab, you must compare it with the literature value and if possible, calculate a % error.

	\subsection*{Evaluation of Method}
	% How confident are you in the results?  Include three weakness and three strengths and again avoid being superficial.
	% Many students fade at the end of their reports and it costs them one grade level.

	\subsection*{Further Investigations}
	% Suggest two other experiments that could be done to enhance or further extend what you have done.
	% Give this thought and avoid being superficial.

	\section*{Bibliography \& References}
	% Use in-text citations for background and when referencing figures

	\newpage
	\setcounter{page}{1}
	\pagenumbering{Roman}
	\listoftables
\end{document}
