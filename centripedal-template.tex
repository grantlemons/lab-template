% Template by Grant Lemons
\documentclass{lab}

% Document Info
\newcommand{\name}{Grant Lemons}
\newcommand{\reporttitle}{Effect of various variables on the speed of rotational motion}

\begin{document}
     \insertTitlePage

    \section{Background}
	% stuff
%------------------------------------------------------------------------------------------------------------------------------
	\section{Research Questions}
        \begin{itemize}
                \vspace{-0.2cm}
                \itemsep-3pt
                \item How does the the radius of the rotational motion effect the rotational speed of the bob?
                \item How does the mass of the bob effect the rotational speed of the bob?
                \item How does the mass of the counterweight effect the rotational speed of the bob?
        \end{itemize}
	
	\section{Hypotheses}
        \begin{itemize}
                \vspace{-0.2cm}
                \itemsep-3pt
                \item I expect that when increasing the radius of the rotation, the rotational speed of the bob will decrease and vice versa.
                \item I expect that when increasing the mass of the bob, the rotational speed will decrease and vice versa.
				\item I expect that increasing the mass of the counterweight will increase the rotational speed and vice versa.
        \end{itemize}
%------------------------------------------------------------------------------------------------------------------------------
	\section{Tables of Variables}
	\centering
	\noindent\begin{minipage}{\textwidth}
	\subsection{Variables - Radius}
	\vspace{-0.2cm}
	\renewcommand{\arraystretch}{1.1}
        \begin{table}[H]
            \begin{tabular}{l|ll|l}
				\dtoprule
				\textbf{Variable type}					& \textbf{Variable}						& \textbf{Details}		 & \textbf{Units \& Uncertainty}	\\
				Independent Variable					& Radius of the rotational motion		& How will it be changed & Centimeters $\pm \SI{0.05}{\cm}$	\\
				Dependent Variable						& Time to complete 20 revolutions		& How will it be changed & Seconds $\pm \SI{0.01}{\second}$	\\
				\multirow{2}{*}{Control Variables}		& The mass of the counterweight			&						 & Grams $\pm \SI{0.1}{\gram}$		\\
														& The mass of the bob					&						 & Grams $\pm \SI{0.1}{\gram}$		\\
				\dbottomrule
	    \end{tabular}
	    \caption{Table 1 - Variables - Radius of Circular Motion}
	    \label{table: variables1}
	\end{table}
	\end{minipage}

	\noindent\begin{minipage}{\textwidth}
    \subsection{Variables - Mass of Bob}
    \vspace{-0.2cm}
	\renewcommand{\arraystretch}{1.1}
        \begin{table}[H]
            \begin{tabular}{l|ll|l}
				\dtoprule
				\textbf{Variable type}					& \textbf{Variable}								& \textbf{Details}		 & \textbf{Units \& Uncertainty}	\\
				Independent Variable					& Mass of the bob								& How will it be changed & Grams $\pm \SI{0.1}{\gram}$		\\
				Dependent Variable						& Time to complete 20 revolutions				& How will it be changed & Seconds $\pm \SI{0.01}{\second}$	\\
				\multirow{2}{*}{Control Variables}		& The mass of the counterweight					&						 & Grams $\pm \SI{0.1}{\gram}$		\\
														& Radius of the rotational motion				&						 & Centimeters $\pm \SI{0.05}{\cm}$	\\
				\dbottomrule
	    \end{tabular}
	    \caption{Table 2 - Variables - Mass of Bob}
	    \label{table: variables2}
	\end{table}
	\end{minipage}

	\noindent\begin{minipage}{\textwidth}
    \subsection{Variables - Mass of Counterweight}
    \vspace{-0.2cm}
	\renewcommand{\arraystretch}{1.1}
        \begin{table}[H]
            \begin{tabular}{l|ll|l}
				\dtoprule
				\textbf{Variable type}					& \textbf{Variable}						& \textbf{Details}		 & \textbf{Units \& Uncertainty}	\\
				Independent Variable					& Mass of the counterweight				& How will it be changed & Grams $\pm \SI{0.1}{\gram}$		\\
				Dependent Variable						& Time to complete 20 revolutions		& How will it be changed & Seconds $\pm \SI{0.01}{\second}$	\\
				\multirow{2}{*}{Control Variables}		& The mass of the bob					&						 & Grams $\pm \SI{0.1}{\gram}$		\\
														& Radius of the rotational motion		&						 & Centimeters $\pm \SI{0.05}{\cm}$	\\
				\dbottomrule
	    \end{tabular}
	    \caption{Table 3 - Variables - Mass of Counterweight}
	    \label{table: variables3}
	\end{table}
	\end{minipage}
%------------------------------------------------------------------------------------------------------------------------------
	\section{Raw Data}
	\noindent\begin{minipage}{\textwidth}
    \subsection{Raw Data - Radius}
    \vspace{-0.2cm}
	\renewcommand{\arraystretch}{1.1}
        \begin{table}[H]
            \centering
            \begin{tabular}{l|llll}
		    	\dtoprule
				Trial		& Counterweight Mass $\pm \SI{0.01}{\gram}$		& Bob Mass $\pm \SI{0.1}{\gram}$		& Radius $\pm \SI{0.05}{\cm}$		& Time of 20 rotations $\pm \SI{0.01}{\second}$	\\
				Trial 1		& 100.75										& 44.2									& 80.0								& 19.24											\\
				Trial 2		& 100.75										& 44.2									& 80.0								& 19.11											\\
				Trial 3		& 100.75										& 44.2									& 80.0								& 19.44											\\
				\hline
				Trial 1		& 200.4											& 44.2									& 80.0								& 16.21											\\
				Trial 2		& 200.4											& 44.2									& 80.0								& 16.33											\\
				Trial 3		& 200.4											& 44.2									& 80.0								& 14.92											\\
				\hline
				Trial 1		& 301.4											& 44.2									& 80.0								& 12.20											\\
				Trial 2		& 301.4											& 44.2									& 80.0								& 13.60											\\
				Trial 3		& 301.4											& 44.2									& 80.0								& 12.13											\\
	    	    \dbottomrule
	    \end{tabular}
	    \caption{Table 4 - Raw Data - Radius of Circular Motion}
	    \label{table: raw-data1}
	\end{table}
	\end{minipage}

	% Raw Data table
	\noindent\begin{minipage}{\textwidth}
    \subsection{Raw Data - Mass of Bob}
    \vspace{-0.2cm}
	\renewcommand{\arraystretch}{1.1}
        \begin{table}[H]
            \centering
            \begin{tabular}{l|llll}
		    	\dtoprule
				Trial		& Counterweight Mass $\pm \SI{0.01}{\gram}$		& Bob Mass $\pm \SI{0.1}{\gram}$		& Radius $\pm \SI{0.05}{\cm}$		& Time of 20 rotations $\pm \SI{0.01}{\second}$	\\
				Trial 1		&&&&\\
				Trial 2		&&&&\\
				Trial 3		&&&&\\
				\hline
				Trial 1		&&&&\\
				Trial 2		&&&&\\
				Trial 3		&&&&\\
				\hline
				Trial 1		&&&&\\
				Trial 2		&&&&\\
				Trial 3		&&&&\\
	    	    \dbottomrule
	    \end{tabular}
	    \caption{Table 5 - Raw Data - Mass of Bob}
	    \label{table: raw-data2}
	\end{table}
	\end{minipage}

	% Raw Data table
	\noindent\begin{minipage}{\textwidth}
    \subsection{Raw Data - Mass of Counterweight}
    \vspace{-0.2cm}
	\renewcommand{\arraystretch}{1.1}
        \begin{table}[H]
            \centering
            \begin{tabular}{l|llll}
		    	\dtoprule
				Trial		& Counterweight Mass $\pm \SI{0.01}{\gram}$		& Bob Mass $\pm \SI{0.1}{\gram}$		& Radius $\pm \SI{0.05}{\cm}$		& Time of 20 rotations $\pm \SI{0.01}{\second}$	\\
				Trial 1		&&&&\\
				Trial 2		&&&&\\
				Trial 3		&&&&\\
				\hline
				Trial 1		&&&&\\
				Trial 2		&&&&\\
				Trial 3		&&&&\\
				\hline
				Trial 1		&&&&\\
				Trial 2		&&&&\\
				Trial 3		&&&&\\
	    	    \dbottomrule
	    \end{tabular}
	    \caption{Table 6 - Raw Data - Mass of Counterweight}
	    \label{table: raw-data3}
	\end{table}
	\end{minipage}
%------------------------------------------------------------------------------------------------------------------------------
	% Results / Processed Data table	
	\noindent\begin{minipage}{\textwidth}
    \section{Calculated Data}
    \vspace{-0.2cm}
	\renewcommand{\arraystretch}{1.1}
        \begin{table}[H]
            \centering
            \begin{tabular}{l|l|l|l}
				\dtoprule
				% The styling of this depends on your data
				% You just kind of need to figure it out
				% The ampersand (&) seperates values on the same row
				% The letters (in this case l|l|l|l) show how many rows there are and the alignment (left)
				% Use \\ at the end of every row
	    	    \dbottomrule
	    \end{tabular}
	    \caption{Table 7 - Calculated Data}
	    \label{table: calculated-data}
	\end{table}
	\end{minipage}
%------------------------------------------------------------------------------------------------------------------------------
	\section{Analysis}

	\noindent\begin{minipage}{\textwidth}
    \vspace{-0.2cm}
	\renewcommand{\arraystretch}{1.1}
        \begin{table}[H]
            \centering
            \begin{tabular}{L{1.445in}|L{1.445in}|L{1.445in}|L{1.445in}}
            \dtoprule
            \textbf{Error or Problem}    & \textbf{Type of Error}    & \textbf{Error's Effect}    & \textbf{Suggestion for Improvement}\\
            \hline
            Artificial increase in recorded temperature from stirring the thermometer & Systematic & The change in temperature calculated was less than it should have been & Be consistent with stirring for each temperature recorded \\
            \hline
            Placement of the thermometer & Random & Recorded temperature does not reflect the average temperature & Stir the water while it heats, such that the heated water is distributed more evenly \\
            \hline
            Different initial temperatures & Systematic & The change in temperature in the second trial was less than it should have been & Use new, room temperature, water for each trial \\
            \dbottomrule
        \end{tabular}
	\caption{Table 8 - Errors \& Suggestions}
	\label{table: errors}
	\end{table}
	\end{minipage}

	\section{Conclusion}
	% Use your data and the results to formulate, summarize and support a conclusion.
	% Use evidence-based reasoning in your writing by referencing your data, results tables and graphs as proof of what is claimed.
	% Your conclusion(s) should be clearly related to the research question and the purpose and stated hypothesis.
	% If a numerical result is the object of the lab, you must compare it with the literature value and if possible, calculate a % error.

	\subsection{Evaluation of Method}
	% How confident are you in the results?  Include three weakness and three strengths and again avoid being superficial.
	% Many students fade at the end of their reports and it costs them one grade level.

	\newpage
	\setcounter{page}{1}
	\pagenumbering{Roman}
	\listoftables
\end{document}

% Experiment 1
% mass +/- 0.01g Time +/- 0.01s length +/- 0.05cm	stopper mass  +/- 0.1g
% 100.75    & 19.24   & 80    & 44.2
% 100.75    & 19.11   & 80    & 44.2
% 100.75    & 19.44   & 80    & 44.2
% 200.4     & 16.21   & 80    & 44.2
% 200.4     & 16.33   & 80    & 44.2
% 200.4     & 14.92   & 80    & 44.2
% 301.4     & 12.2    & 80    & 44.2
% 301.4     & 13.6    & 80    & 44.2
% 301.4     & 12.13   & 80    & 44.2
% 
% Experiment 2
% 100.75    & 16.14    & 50     & 44.2
% 100.75    & 16.25    & 50     & 44.2
% 100.75    & 17.24    & 50     & 44.2
% 100.75    & 23.85    & 100    & 44.2
% 100.75    & 23.14    & 100    & 44.2
% 100.75    & 22.35    & 100    & 44.2
% 
% Experiment 3
% 201.5    & 16.05    & 80    & 44.2
% 201.5    & 16.32    & 80    & 44.2
% 201.5    & 15.99    & 80    & 44.2
% 201.5    & 18.95    & 80    & 94
% 201.5    & 18.31    & 80    & 94
% 201.5    & 19.78    & 80    & 94
